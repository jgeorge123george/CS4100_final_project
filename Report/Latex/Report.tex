% One really important thing to note is that Overleaf is simply an online product that enables collaboration for LaTeX files. You may choose to create a LaTeX environment locally on your machine (although...it takes >10 GB of storage). To do the latter, refer to: https://www.latex-project.org/get/
% To learn more about how to use LaTeX and its syntax, check out the Overleaf documentation: https://www.overleaf.com/learn

% Defining a document class is the first line in a LaTeX file. It sets the overall layout and styling of the document.
\documentclass[a4paper, twocolumn]{article}

% These packages enable customization of various aspects of the document. Think of them like "import" statements in Python.
\usepackage{amsmath}
\usepackage{graphicx}
\usepackage{geometry}
\usepackage{floatrow}
\usepackage{layout}
\usepackage{amssymb} 
\usepackage{multirow}
\usepackage{caption}
\geometry{margin=1in}
\usepackage{authblk}
\usepackage{indentfirst}
\usepackage[hidelinks]{hyperref}

% \providecommand defines a new command in LaTeX if it does not exist. This is particularly helpful to do if you find yourself repeating a specific set of commands for your a template.
\providecommand{\keywords}[1]
{
  \small	
  \textbf{\; \textit{Keywords---}} #1
}

% Anything between \begin{document} and \end{document} is what gets printed on the page
\begin{document}

\title{\textbf{\huge{Insert Fancy Project Title}}}
\author{\textbf\large{Joseph George, Tammer Haddad}}
\affil{\textbf{Northeastern University, Boston, 02115}}
\date{\today}

% This is required to print the above 4 lines
\maketitle

% Optional, but useful default command in LaTeX to compose an Abstract section.
\begin{abstract}
Your abstract goes here. A well-written abstract introduces the problem and motivation (briefly), highlights the approach taken by the authors, and summarizes the results. The idea is to give the readers an overview of what to expect from the paper.
\end{abstract}\maketitle

\keywords{one, two, three, four}

% \section command is a section title's style. By default, they are numbered. If you don't want them numbered, add a * like below.
\section*{Introduction}

The introduction section serves to expand upon the motivation, and to contextualize the specific problem chosen by the authors to work upon. In the absence of a dedicated related work section, comparison to previous approaches may be highlighted here. The section discusses the approach at a high-level, often with supporting figures where appropriate. The following is an example of how to insert an image in Latex.

% How to add figures! Notice the [h] command. This is an argument for positioning the figure as you want. The default is that the LaTeX processor chooses where to place the image in the entire document
% To learn more about positioning, read the Positioning section on OverLeaf https://www.overleaf.com/learn/latex/Inserting_Images
\begin{figure}[h]
    \centering
    \includegraphics[width=\linewidth]{batman.jpg}
    \caption{A cool graph. This image does not reflect the author's allegiance to the DC Comics universe.}
    % Labels are helpful to refer to the figure, table, equation, etc. within the document. As expected, it is far easier to remember a label name and the numerical value it was in.
    \label{fig:1}
\end{figure}

Here, I refer to Figure \ref{fig:1} without actually listing the actual figure number in plaintext!

\section{Example Section}
This is an example of a numbered section. Sections typically seen in a research paper are Related Work, Preliminaries, Data, Methodology, Experiments, Results, and Discussion. The order in which these sections appear may depend on the venue.

\subsection{Example Subsection}
This is an example of a subsection. For example, Definitions could be a subsection of the Preliminaries section.

\section{Formulae, Tables}
% Here's a cheatsheet for LaTeX symbols: https://www.cmor-faculty.rice.edu/~heinken/latex/symbols.pdf

\subsection{Typing Equations in \LaTeX}
This subsection demonstrates the use of inline formulae, like $e=mc^2$, or equations like the one below on a separate line. \[\int_a^b x^2\;\mathrm{d}x= \tfrac{1}{3} x^3 \Big|_a^b\]

For numbered equations, we use the \textit{equation} environment:

\begin{equation}
    \int_a^b x^2\;\mathrm{d}x= \tfrac{1}{3} x^3 \Big|_a^b
\end{equation}

\subsection{Tables}
This subsection demonstrates the use of tables in a \LaTeX document. Resources like \href{https://www.tablesgenerator.com/}{\texttt{tablesgenerator.com}} can be used to generate code for tables like the one below.\

% Same rules as Figures above apply to Tables in LaTeX!
% More on table formatting: https://www.overleaf.com/learn/latex/Tables
\begin{table}[h]
    \centering
    \begin{tabular}{ |p{3cm}||p{1.2cm}|p{1.2cm}|p{1.2cm}|  }
     \hline
     \multicolumn{4}{|c|}{Country List} \\
     \hline
     Country Name     or Area Name& ISO ALPHA 2 Code &ISO ALPHA 3 Code&ISO numeric Code\\
     \hline
     Afghanistan   & AF    &AFG&   004\\
     Aland Islands&   AX  & ALA   &248\\
     Albania &AL & ALB&  008\\
     Algeria    &DZ & DZA&  012\\
     American Samoa&   AS  & ASM&016\\
     Andorra& AD  & AND   &020\\
     Angola& AO  & AGO&024\\
     \hline
    \end{tabular}
    \caption{A Random Table}
    \label{tab:my_label}
\end{table}

% Double slash \\ forces a new line in LaTeX.
Here, I refer to Table \ref{tab:my_label} without actually listing the actual Table number!\\

References to books \cite{DUMMY:1} or articles \cite{ARTICLE:1} can be made like this, with the full citations defined in a \emph{.bib} file.

\section{Github Link \& Team Contributions}
Please include a link to your GitHub repo and a 65-word (total) statement summarizing each member's contributions.

\bibliography{references}
\bibliographystyle{acm}

\end{document}